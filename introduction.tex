\pagebreak %début sur une nouvelle page
\section{Introduction}
	
	\paragraph{}
	Le stage de fin d'études à \textit{Grenoble INP - Phelma} est un stage ingénieur de recherche et d'application. D'une durée minimale de cinq mois, ce stage permet l'intégration d'un projet sur une longue période. Il peut aussi amener à la réalisation d'un projet depuis la rédaction d'un cahier des charges jusqu'à la création d'un produit dans un milieu professionnel. En conséquence, ce stage est un premier pas vers le monde de l'entreprise et offre la possibilité de finaliser son projet professionnel. 
	
	\paragraph{}
	Ce stage a été réalisé au sein de \textit{SOLUTEC}, une entreprise de service du numérique (ESN). Les entreprises de ce secteur d'activité vivent un plein essor depuis le début des années 2000. En effet, les grandes entreprises ayant des infrastructures de plus en plus importantes et un besoin d’applications spécialisées croissant, font appellent aux ESN qui comblent se besoin en intervenant directement au sein des équipes.
	Les domaines que couvre \textit{SOLUTEC} sont axés sur l'informatique de gestion au sein de grandes entreprises. Cependant, la structure du \textit{lab'SOLUTEC} permet aux étudiants ingénieur stagiaires de réaliser des projets indépendants liés à des technologies innovantes.

	\paragraph{}
	Dans ce stage, l'objectif est la création d'un application Android intégrant deux technologies d'actualité: la réalité augmentée et la réalité virtuelle. Initialement abordé par un client, ce sujet a été abandonné et proposer en tant que stage de manière indépendante. L'application traite d'un problème de gestion de matériels au sein des entreprises. Elle doit permettre la reconstruction du plan d'une pièce, le référencement du matériel qu'elle contient et le suivi de ce matériel dans le temps par des notes de maintenance par exemple.\\
	Ces technologies n'ayant jamais été abordées au sein de l'entreprise, ce projet est effectué dans un but de recherche et de veille technologique.
	L'application a été réalisé en groupe de quatre personnes et entièrement développée en JAVA grâce à l'univers de développement \textit{Android Studio}.

	\paragraph{}
	Tout d'abord, le sujet impose la reconstruction du plan d'une pièce en deux dimensions au moyen d'un smartphone, c'est la raison pour laquelle les recherches sur ce point sont abordées en première partie de ce document.
	Ayant été particulièrement impliqué dans cette partie elle sera présentée en détails dans les parties suivantes, articulés autour du traitement de données des capteurs, de la détection d'objet dans une image et aussi de l'algorithme de reconstruction d'une pièce.\\
	Enfin, l'organisation et la mise en place de la méthode AGILE tout au long du projet sera présentée en dernière partie.